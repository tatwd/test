\documentclass{beamer}

\usepackage{multicol}

% Warsaw
\usetheme[progressbar=frametitle]{metropolis}
\setbeamertemplate{frame numbering}[fraction]
\useoutertheme{metropolis}
\useoutertheme{metropolis}
\usefonttheme{metropolis}
\usecolortheme{spruce}
\setbeamercolor{background canvas}{bg=white}

\definecolor{mygreen}{rgb}{.125,.5,.25}
\usecolortheme[named=mygreen]{structure}
%\usecolortheme{crane}

\title[Short Title]{Functions, Limits, Derivatives}
%\subtitle{Subtitle Here}
\author[Jaron]{Jaron King}
\institute{\textbf{Institute}: this is my first slide made by \LaTeX}
\date{}

\setbeamercovered{transparent=15}

\begin{document}
\metroset{block=fill}

\begin{frame}
\titlepage
\end{frame}

%\begin{frame}
%Hello
%\end{frame}

\begin{frame}[t]{Functions}\vspace{4pt}

\begin{block}{Definition of a Function}
\vspace{0.5em}
A \textbf{function} $f$ is a rule that assigns to each element $x$ in a 
set $D$ exactly one element, called $f(x)$, in a set $E$. 
\vspace{0.5em}
\end{block}

\vspace{10pt}
Set $D$ is called the 
\only<1>{\line(1,0){50}}
\only<2>{\textcolor{red}{domain}}
\, of the function.\\[10pt]
 
Set $E$ is called the 
\only<1>{\line(1,0){50}}
\only<2>{\textcolor{red}{range}}
\, of the function.

\end{frame}

\begin{frame}{Your are Very First Flash Card}\vspace{10pt}
\begin{columns}[onlytextwidth]

\column{0.4\textwidth}
$\sqrt{x^2}=$\\[10pt]
\begin{enumerate}[(A)]
\item $x$
\item $-x$
\item $|x|$
\item undefined
\end{enumerate}

\column{0.6\textwidth}
\only<2>{
$\sqrt{x^2}=
\begin{cases}
-x, & x<0 \\
x, & x \geq\ 0
\end{cases}
$\\[10pt]
}
%\only<2>{
%xx
%}
\end{columns}
\end{frame}

\begin{frame}[t]{Parent Functions}\vspace{4pt}
You should be able to identify by name and sketch a gra[h of each pf the following parent functions.
\begin{enumerate}
\begin{multicols}{3}
\item $y=x$
\item $y=|x|$
\item $y=x^2$
\item $y=x^3$
\item $y=x^b$

\onslide<2->{
\item $y=\sqrt{x}$
\item $y=\sqrt[3]{x}$
\item $y=\frac{1}{x}$
\item $y=2^x$
\item $y=e^x$
}

\onslide<3->{
\item $y=\ln x$
\item $y=\frac{1}{1+e^{-x}}$
\item $y=\sin x$
\item $y=\cos x$
\item $y=\tan x$
}
\end{multicols}
\end{enumerate}
\end{frame}

\begin{frame}[standout]
\flushleft
Homework: p.342 \#7-21
\end{frame}

\end{document}